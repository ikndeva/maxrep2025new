%% notation.tex

%% tmp: modelfull
\newcommand{\FY}{(\Sigma,Y)}
\newcommand{\pred}{pred}
\newcommand{\etal}{\textit{et al.}}
\newcommand{\dtcano}{\op{dt-cano}}
\newcommand{\ADTC}{\textsc{ADTC}}
\newcommand{\minWHS}{\textsc{minWHS}}
%% \newcommand{\dtcano}{dtcano}


%%
\newcommand{\medstrut}{\rule{0pt}{1pc}}
\newcommand{\iFor}[2]{\textbf{for} {#1} \textbf{do}\hspace{0.25em}{\relax #2}}%exp

% 2nd version
%% basic 
\newcommand{\sig}[1]{\mathcal{#1}}
\newcommand{\alg}[1]{\idsc{#1}} %% algorithm 
\newcommand{\proc}[1]{\alg{#1}}
\newcommand{\op}[1]{\texttt{#1}} %% operation 
\newcommand{\pb}[1]{{\mbox{\textsc{#1}}}} %% problem name
%% \newcommand{\pb}[1]{\mbox{\textsc{#1}}} %% problem name
%% \newcommand{\pb}[1]{\textsc{#1}} %% problem name

%%
\newcommand{\st}[1]{\substack{#1}}

%% notation
%%%% algebra
\newcommand{\R}{\sig R} %% (semi)ring 
\newcommand{\B}{\mathbb{B}}
\newcommand{\M}{\sig M} %% monoid 
\newcommand{\T}{\sig T} %% tensor
\newcommand{\Vol}{\idrm{Vol}} %% tensor
\newcommand{\TT}{\mathbb{T}} %% tensor
%%%%
\newcommand{\SemiringForm}{\R = (R, \add, \mul, e^\add, e^\mul)}
\newcommand{\SemiringBody}{(R, \add, \mul, e^\add, e^\mul)}
\newcommand{\MultiSemiringForm}[1][\Sigma]{(R, \add, \set{\mulsym[\sigma]}_{\sigma \in #1}, e^\add, e^\mul)} %% notation
%%%% set product 
\newcommand{\prodidx}[4]{\prod\idx{#2}{#3}{#4} {#1}_{#2}} 
\newcommand{\prodseq}[4]{{#1}_{#3}\kern-.5pt\times\dots\times\kern-.5pt{#1}_{#4}}
%%%% tuple
\newcommand{\tupidx}[4]{({#1}_{#2})\idx{#2}{#3}{#4}} 
\newcommand{\tupseq}[4]{({#1}_{#3}, \dots, {#1}_{#4})}
%%%
\newcommand{\domprod}[1]{{#1}_{\times}}
%\newcommand{\Prod}{{\scriptsize\prod}\kern-1pt}
\newcommand{\Prod}{\raisebox{-.99pt}{\scalerel*{\prod}{\overline{M}}}\kern0pt}
%\scalerel
\newcommand{\prodvec}[1]{{\vec{#1}}_{\Pi}}
\newcommand{\Prodvec}[1]{\Prod\vec{#1}}


%%%% classes

%%
\newcommand{\add}{\oplus}
\newcommand{\mul}{\otimes}
\newcommand{\bigadd}{\bigoplus}
\newcommand{\bigmul}{\bigotimes}
\newcommand{\setmn}{\!-\!}
\newcommand{\eaddunit}{e^\oplus}
\newcommand{\emulunit}{e^\otimes}
%% 
\newcommand{\mulsym}[1][\sigma]{\mul^{(#1)}}
\newcommand{\mulset}{\mulsym[\uplus]}
\DeclareMathOperator*{\bigmulset}{\mbox{$\mulsym[\uplus]$}}% for set
%\DeclareMathOperator*{\mulset}{\mbox{$\bigmul^{(\uplus)}$}}% for set

%%% complexity
\newcommand{\wname}[2][\R]{\idrm{#2}({#1})} %% weighted name
\newcommand{\NPR}[1][\R]{\idrm{NP}(#1)}
\newcommand{\SRTM}[1][\R]{\idrm{SRTM}(#1)}
\newcommand{\EMT}{\idsc{EMT}}

%%% problems: ATDC 
\newcommand{\Data}{\pow\D}
\newcommand{\Feat}{\pow\F}

%%% problems
\newcommand{\SumProd}{\textsc{SumProd}}
%% \newcommand{\NP}[1]{\idrm{NP}{#1}}
%% \newcommand{\NPR}[1][(\R)]{\idrm{NP}{#1}}

% 1st version
%% \renewcommand{\vec}[1]{\boldsymbol{#1}} %% loaded by amsbmy (amsmath)
\newcommand{\vvec}[1]{\overrightarrow{#1}} 
\renewcommand{\-}{\textrm{-}}

%% \newcommand{\DT}[1][d]{\mathcal{DT}^{#1}} 
\newcommand{\DT}[1][d]{\id{DT}^{#1}} 
\newcommand{\D}{\sig D}
\renewcommand{\L}{\sig L}
\newcommand{\HOMEXT}{\idsc{Homext}}

\newcommand{\X}{\sig X}
\newcommand{\Y}{\sig Y}
\newcommand{\Z}{\sig Z}
\newcommand{\F}{\mathbf{\Sigma}}
\renewcommand{\S}{\sig S}
\renewcommand{\H}{\sig H}

\newcommand{\Lv}{\idrm{Lv}}
\newcommand{\dep}{\op{dep}}
\newcommand{\size}{\op{size}}
\newcommand{\err}{\op{err}}
\newcommand{\supp}{\op{supp}}
%% \newcommand{\relerr}{\op{relerr}} %
\newcommand{\tp}{\op{tp}}
\newcommand{\fp}{\op{fp}}
\newcommand{\tn}{\op{tn}}
\newcommand{\fn}{\op{fn}}
\newcommand{\disag}{\op{disag}}
\newcommand{\fs}{\op{fset}} %

%% Unfairness range 
\newcommand{\infrac}[2]{({#1}/{#2})}
\newcommand{\percent}{\%}

%% math 

\newcommand{\nd}[1]{\langle #1\rangle} %node 
\renewcommand{\=}{\kern-2pt=\kern-2pt}
\newcommand{\mn}[1]{_{-#1}}
\newcommand{\spl}[1]{_{|#1}}


%% math
\newcommand{\dom}{\op{dom}}
\newcommand{\rng}{\op{range}}

\newcommand{\ind}[1]{\indone[#1]}
\newcommand{\Ind}[1]{\indone\kern-0.125em\left[\,\rule{0pt}{.9em} #1\,\right]}
\newcommand{\indone}{\mathbbm{1}}

%%  tensor 
%%  constant 
\newcommand{\zero}{\mathbbm{0}}
\newcommand{\one}{\mathbbm{0}}
\newcommand{\met}[1][]{\vec\mu #1}

%% whiteboard digits
\newcommand{\Ttensor}[1][]{\mathbb{T}[\vec\M, \R]}
\newcommand{\Tzero}{{\mathbbm{O}}}
\newcommand{\Tone}{{\mathbbm{1}}}
% \newcommand{\Tzero}{\midsym{\mathbbm{O}}}
% \newcommand{\Tone}{\midsym{\mathbb{1}}}

%% box oper
\newcommand{\bigboxplus}{\mybigop{\boxplus}}
\newcommand{\bigboxtimes}{\mybigop{\boxtimes}}
\newcommand{\sym}[1]{\overline{#1}}

%% Oh oper
\newcommand{\bigoplusz}{\mybigop{\oplus}}
\newcommand{\bigotimesz}{\mybigop{\otimes}}
\DeclareMathOperator*{\mybigoplus}{\bigoplusz}
\DeclareMathOperator*{\mybigotimes}{\bigotimesz}

%% tensor operations
\newcommand{\Tplus}{\mathbin{\widetilde{\oplus}}}
\newcommand{\Ttimes}{\mathbin{\widetilde{\circledast}}}

% \newcommand{\Tplus}{\mathrel{\ooalign{\hss$\oplus$\hss\cr\hss$\bigcirc$\hss}}}
% \newcommand{\Ttimes}{\mathrel{\ooalign{\hss$\circledast$\hss\cr\hss$\bigcirc$\hss}}}

\newcommand{\bigtenplusz}{\mybigop{\Tplus}}
\newcommand{\bigtentimesz}{\mybigop{\Ttimes}}
\DeclareMathOperator*{\bigtenplus}{\bigtenplusz}
\DeclareMathOperator*{\bigtentimes}{\bigtentimesz}

\newcommand{\bby}{\!\times\!}

% %% tensor operations
% \newcommand{\Tplus}{\mathrel{\ooalign{\hss$\oplus$\hss\cr\hss$\bigcirc$\hss}}}
% \newcommand{\Ttimes}{\mathrel{\ooalign{\hss$\circledast$\hss\cr\hss$\bigcirc$\hss}}}

% \newcommand{\bigtenplusz}{\mybigop{\Tplus}}
% \newcommand{\bigtentimesz}{\mybigop{\Ttimes}}
% \DeclareMathOperator*{\bigtenplus}{\bigtenplusz}
% \DeclareMathOperator*{\bigtentimes}{\bigtentimesz}

\newcommand{\poly}{\,\idrm{poly}}
\newcommand{\stm}[1][.9em]{\rule{0pt}{#1}} %%mystem

%%  tentative : slice operator
\newcommand{\Dint}[2]{[#1\hskip1pt\raise-0pt\hbox{$:$\hskip1pt}#2]}
% %%% tex hack: "::"コマンドを作成する
% %%% https://tex.stackexchange.com/questions/4216/how-to-typeset-correctly\mathchardef\ordinarycolon\mathcode`\.
% \mathcode`\.=\string"8000
% \begingroup \catcode`\.=\active
%   \gdef.{\hbox{\hskip1pt\raise-0pt\hbox{$:$\hskip1pt}}}
% % \gdef.{\hbox{\hskip1pt\raise-2pt\hbox{$\cdot$\hskip1pt}}}
% \endgroup
% %%% end tex hack

%% MPF: explanation
\newcommand{\proj}[2]{_{|#1_{#2}}}

%% Algorithms
%\newcommand{\EMTUC}{\textsf{UncEMT}}

\newcommand{\CT}{\mathbbm{CT}}



% cref
\crefname{theorem}{Theorem}{Theorems}
\crefname{cor}{Corollary}{Corollaries}
\crefname{proposition}{Proposition}{Propositions}
\crefname{lemma}{Lemma}{Lemmas}
\crefname{property}{Property}{Properties}
\crefname{condition}{Condition}{Conditions}
\crefname{fact}{Fact}{Facts}
\crefname{claim}{Claim}{Claims}
\crefname{problem}{Problem}{Problems}
\crefname{remark}{Remark}{Remarks}
%
\crefname{algorithm}{Algorithm}{Algorithms}
%
\crefname{section}{Sec.}{Secs.}
\crefname{subsection}{Sec.}{Secs.}
%
\crefname{proof}{Proof}{Proofs}



%% EOF
% %\newcommand{\circledplusz}{\mathrel{\ooalign{\hss$\oplus$\hss\cr\hss$\circ$\hss}}}
% %\newcommand{\circledoplus}{\mathrel{\ooalign{\hss$+$\hss\cr\hss$\bigcirc$\hss}}}
% \newcommand{\tcircledast}{\mathrel{\ooalign{\hss$\circledast$\hss\cr\hss$\bigcirc$\hss}}}
% \newcommand{\tcircleoplus}{\mathrel{\ooalign{\hss$\oplus$\hss\cr\hss$\bigcirc$\hss}}}
% %\newcommand{\circleddast}{\mathrel{\ooalign{\hss$\textcircled{$\ast$}$\hss\cr\hss$\bigcirc$\hss}}}
% %\newcommand{\circledoplus}{\mathrel{\ooalign{\hss$\textcircled{+}$\hss\cr\hss$\bigcirc$\hss}}}

