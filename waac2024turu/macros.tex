%%% 数式モードで使う表記
\newcommand{\set}[1]{\{#1\}}
\newcommand{\iffdef}{\stackrel{\idrm{def}}{\iff}}
\renewcommand{\mod}{\idrm{mod}}
\newcommand{\Key}[1]{\Statex\hspace{-1.2\leftmargin}  \textbf{#1}\hspace{.25em}}
\newcommand{\Proc}[1]{\Key{Procedure}{#1}}
%\newcommand{\Proc}[1]{\Statex\hspace{-1.2\leftmargin}  \textbf{Procedure}\hspace{.25em}{#1}}
\newcommand{\ename}[2]{\textbf{#1} (\textit{#2})}
\newcommand{\itl}[1]{\textit{#1}}
\newcommand{\sig}[1]{\mathcal{#1}}

%%% 数式モードで文字が開く問題を解消する機能

\def\id#1{\ensuremath{\mathit{#1}}}
\let\idit=\id
\def\idbf#1{\ensuremath{\mathbf{#1}}}
\def\idrm#1{\ensuremath{\mathrm{#1}}}
\def\idtt#1{\ensuremath{\mathtt{#1}}}
\def\idsf#1{\ensuremath{\mathsf{#1}}}
\def\idcal#1{\ensuremath{\mathcal{#1}}} 

%%% 参照の便利版
\usepackage{cleveref}
\crefname{algorithm}{アルゴリズム}{Algorithms}
\crefname{table}{表}{Tables}
\crefname{figure}{図}{Figures}
\crefname{section}{節}{Sections}
\crefname{chapter}{章}{Chapters}
\crefname{problem}{問題}{Problems}
\crefname{definition}{定義}{Definitions}
\crefname{lemma}{補題}{Lemmas}
\crefname{fact}{事実}{Facts}
\crefname{property}{性質}{Properties}
\crefname{proposition}{命題}{Propositions}
\crefname{thmeorem}{定理}{Theorems}
\crefname{cor}{系}{Corollaries}
\crefname{footnote}{脚注}{Footnotes}