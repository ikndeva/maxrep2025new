%% intro2b.tex
%% intro2 => shorten to 300 word
\section{Introduction (re-trimed)}

\subsection{Background}
Full-text search, the process of finding all occurrences of a given pattern within a text, is a fundamental task in information retrieval and sequence analysis. Text indexes, pre-computed data structures, significantly accelerate search operations. The \textit{suffix tree}, the most basic \textit{graph-based text index}, efficiently stores all suffixes of a text in linear space and can be constructed in linear time. It enables the efficient solution of a wide range of text search and sequence analysis problems. However, graph-based indexes suffer from limitations such as high memory usage and poor cache locality due to their complex, pointer-based structures.

\textit{Array-based text indexes}, such as the \textit{suffix array} (SA)~\cite{manber:myers1993suffixarrays} and the \textit{Burrows-Wheeler Transform} (BWT) array~\cite{Ferragina05:FM}, have emerged as space-efficient alternatives to graph-based indexes. Despite having simpler structures, typically arrays of integers or characters, they retain sufficient information for pattern searching when combined with appropriate auxiliary data structures. Their simplicity and efficiency have made them popular in large-scale text applications like gene sequence analysis.

However, array-based indexes lose much of the structural information present in the suffix tree, hindering the efficient solution of complex sequence analysis problems. While extensive research has been conducted to address this limitation, existing array-based algorithms for many problems have not yet achieved the $O(e_R)$-time complexity of algorithms realized by graph-based text indexes, where $O(e_R)$ represents the number of edges in the corresponding graph structure.

\subsection{Research Goal}

This study focuses on developing efficient methods for enumerating maximal repeats (MRs) within a text using array-based text indexes. MRs are fundamental string features, characterizing many other string properties. Their efficient enumeration is crucial for various string analysis tasks. We assume a scenario where the text is preprocessed into an index data structure to support efficient operations and minimize construction time in the context of multiple analyses.

Graph-based indexes, particularly the \textit{Compact Directed Acyclic Word Graph} (CDAWG)~\cite{blumer1985smallest}, can compactly represent all MRs as a set of nodes in $O(e_R+e_L)$ space (see Raffinot~\cite{raffinot2001maximal}), where $O(e_R)$ and $O(e_L)$ represent the numbers of forward and backward edges in the CDAWG, respectively. This allows for MR enumeration in $O(e_R)$ or $O(e_L)$ time, the fastest known result for this problem.

In contrast, existing enumeration algorithms for maximal repeats can be classified into two categories based on the type of graph index traversed and the direction of traversal (see Ohlebusch~\cite{ohlebusch2013bookbioinfo}): \textit{Type-$1$} (bottom-up suffix tree traversal)~\cite{kasai:lee2001lcp:linear} and \textit{Type-$2$} (top-down Weiner tree traversal)~\cite{abouelhoda2004replacing} algorithms. Both types require $O(n)$ time, where $n$ is the text length. No $O(e_R)$-time algorithm has been previously known for array-based MR enumeration.

\subsection{Main results}

To address this challenge, we propose two novel array-based algorithms, Algorithms A and B, that enumerate all MRs of a text $T$ in $O(e_R)$ time, matching the performance of the fastest graph-based index structure, the CDAWG of $T$.
These algorithms are simple procedures that utilize only standard array indexes and auxiliary structures commonly used in text search. This simplicity facilitates implementation and allows for easy integration with future advancements in array index implementation technology.
By adapting the underlying arrays to various text indexes, including uncompressed, entropically compressed, and repetition-aware compressed indexes, we achieve a wide range of time-space trade-offs for MR enumeration.

\subsection{Techniques}

To obtain the main results, we devise the following techniques. 
\begin{itemize}
\item 
Algorithm A: This algorithm performs top-down traversal of the virtual suffix tree, unlike existing Type-1 algorithms that use bottom-up traversal. It leverages a novel characterization of the region of MRs within the suffix tree and incorporates an O(1)-time left-branching test to prune unnecessary branches.

\item 
Algorithm B: This algorithm extends existing Type-2 algorithms by introducing an efficient procedure to jump directly to other MRs, skipping non-branching nodes. It utilizes a bidirectional index and internal substring matching techniques to achieve $O(e_R)$-time MR enumeration.
\end{itemize}

These results represent significant advancements in MR enumeration using array-based text indexes, achieving $O(e_R)$-time complexity, matching the performance of the fastest graph-based methods.