%%% app.tex
\section{Appendix}
\subsection{Another draft: An abstract, or a short introduction, the 2nd version}

\textbf{Backgrounds and the previous approaches.}
Maximal repeats are one of the most well-studied string features. They play a crucial role in various applications, including text mining and gene analysis. This study investigates efficient enumeration of maximal repeats. While maximal repeats can be enumerated in $O(e)$ time using a data structure called CDAWG, where e is the number of edges in the data structure, this approach is not suitable for large-scale data. Therefore, simpler and more efficient algorithms are needed. 
%% 
%%\textbf{Previous approach.}
Since the 2000s, many simpler and more efficient algorithms using array structures such as suffix arrays and Burrows-Wheeler transform have been proposed. They have also been extended to enumerate various string features other than maximal repeats. However, none of these proposed algorithms can enumerate all maximal repeats in $O(e)$ time, the same as CDAWG. 

\textbf{Contributions of this paper.}
To address this problem, this study proposes two new algorithms that enumerate all maximal repeats in $O(e)$ time by using suffix arrays and Burrows-Wheeler transform, along with some auxiliary arrays. Our contiributions are summarized as follows. 

\begin{enumerate}
\item 
The first algorithm performs a top-down traversal of a virtual suffix tree of the text using a suffix array. To reduce the computation time from construction time to a better order, we use the LCP array and a left maximality verification algorithm combining the suffix array and the reverse suffix array. The key to the algorithm's efficiency is the characterization of vertex placements representing maximal repeats at internal nodes of the suffix tree. This is an interesting application of the well-known suffix property regarding the existence of suffix links in suffix trees. 

\item 
On the other hand, the second algorithm is an extension of the conventional linear-time algorithm. Similar to conventional array-based algorithms, this algorithm traverses a Weiner tree, which is a tree consisting of suffix links of the text, using the Burrows-Wheeler transform. To improve the linear time to output linear time, we use the interval representation, a concise representation of maximal repeats. 
\end{enumerate}

\textbf{Technical overview.}
Specifically, in the second algorithm, we use a subroutine that converts the interval representation of the suffix array of the forward text to that of the reverse text in logarithmic time. This allows us to enumerate all maximal repeats in output linear time with a logarithmic factor. These algorithms are the first to achieve output-proportional time for enumerating maximal repeats using array-based index structures. 

\textbf{Extensions}
Additionally, this study provides an efficient algorithm to obtain the corresponding CDAWG data structure from the array-based index structure in output linear time. Furthermore, we explain how to enumerate various string features such as MAW, MUS, and MRW from the array-based index structure by extending maximal repeats. 

\textbf{Experimental results}
Finally, computational experiments compare the computation time and memory usage of the proposed algorithms with algorithms using suffix trees and Weiner trees.

\medskip 
(End of introduction)

%% EOF

