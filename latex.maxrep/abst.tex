%%% abst.tex
\begin{abstract}
  This paper consider the problem of efficiently enumerating classes of unusual words in a string $S$ based on a combination of a few well-known indexing arrays, namely, the suffix array, LCP array, BWT array equipped with auxiliary structures. In particular, we consider the classes of minimal rare words (MRWs), minimal unique substrings (MUSs), and minimal absent words (MAW), unusual words extensively studied in applications of genetic sequence analysis. 
  Although existing algorithms are effective, they are not output-sensitive, always requiring $\Theta(n)$ time even for a small number of patterns. For the afoementioned classes of patterns, we present output-sensitive algorithms for enumerating patterns in $O(e_R + e_L + occ)$ time and $O(e_L  + \sigma^2 \log n)$ working space, where $occ$ is the number of distinct patterns, and $e_L$ and $e_R$ are parameters, called the numbers of left- and right-extensions of maximal repeats in $S$, respectively, which are upperbounded by $n$, and sublinear, even logarithmic, for highly-repetitive strings. This is the first result to achieve a sublinear time complexity for this problem based on indexing arrays. As the key to our sublinear time complexity, we devise a new simple algorithm to enumerate all distinct maximal repeats in $O(e_L + e_R)$ time using the afoementioned indexing arrays, which will be of independent interest.  
\end{abstract}

%% denote the numbers of left- and right-extensions of maximal repeats in S, respectively, and are upper-bounded by the length n of the string, and can be sublinear in n, up to logarithmic, for highly repetitive strings such as a collection of human genomes.


%%   In this paper, we study efficient enumeration of classes of unusual words in a string.
%%   %% In this paper, we study efficient enumeration of non-trivial MAWs and other classes of unusual words, such as the \textit{minimal unique substrings} (MUSs) and the \textit{minimal rare words} (MRWs) in a string.
%%   A \textit{minimal absent word} (MAW) of a string $S$ is a string $w$ that does not occur in $S$ and any proper substring of $w$ occurs in $S$ as a substring. For any integer $k\ge 0$, as most general form of unusual word, a \textit{minimal rare word} (MRW) of $S$ with frequency $k$ is any string $w$ that occurs in $S$ exactly $k$ times, and any proper substring of $w$ has strictly less occurrences in $S$. An unusual word is said to be non-trivial if it has length at least two.
%% Presently, most practical algorithms for unusual words use the suffix array or the Burrows-Wheeler transform of $S$ rather than space inefficient data structures such as the suffix trees. However, all the previous algorithms for MAWs, MUSs, and MRWs have the total amount of time that grows linearly in the length $n$ of $S$.
%%   In this paper, we present a new algorithm that, based on the suffix and the inverse suffix arrays, longest common prefix array, and an input string, computes the set $MAW(S)$ of all minimal absent words in a string $S$. The algorithm runs in $O(e_R + occ)$, where $occ$ denotes the number $|MAW(S)|$ of solusions, and  $e_R$ denotes the number of right-extensions of maximal repeats in $S$. For any string $S$ of length $n$, $e_R$ is linear in $n$, and can be sublinear in $n$, up to logarithmic.
%%   We also show that modified algorithms computes \textit{minimal unique substrings} (MUSs) and their generalization, called \textit{mininimal rare words} (MRWs), in similar time complexities. 

  
  %% In this paper, we consider enumeration of classes of unusual words, namely, \textit{minimal absent words} (MAW), \textit{minimal unique substrings} (MUSs), and their generalization, called \textit{mininimal rare words} (MRWs). In general, an unusual word is a short string whose frequency in a given string $S$ as substrings is siginificantly deviated from those of its proper subwords. 
  %% The current fastest algorithms for finding such unusual words 
  %% uses as a substrait array-based structures, such as the suffix array or the Burrows-Wheeler transform of $S$ combined with small auxiliary data structures, rather than space inefficient data structures such as the suffix tree.
  %% However, these algorithms are not \textit{output-sensitive} since they always grow linearly in the length $n$ of innput $S$ even when the number of solutions is sublinear in $n$.
  %% In this paper, we present a new algorithm that, based on the suffix array, the inverse suffix arrays, longest common prefix array, and an input string, computes the set $MRW(S)$ of all minimal rare words in a string $S$ in $O(e + occ)$ time, where $occ$ denotes the number $|MRW(S)|$ of solusions, and  $e = e_L + e_R$ denotes the sum of the numbers of left- and right-extensions of maximal repeats in $S$. For any string $S$ of length $n$, $e_R$ is linear in $n$, and can be sublinear in $n$, up to logarithmic, for highly repetitive strings such as collection of human genomes. A key to our algorithm is efficient enumeration of all maximal repeats in a string $S$ in $O(e_R)$ time using the suffix array and other indexing structures based on a novel search strategy. 
  %% We also show how to modify the proposed algorithms for computation of other classes of unusual words.


