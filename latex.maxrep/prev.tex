%%% prev.tex
\section{Previous Algorithms}
\label{sec:prev}

In this section, we review two state-of-the-art algorithm schema, namely \BUSA{} and \TDBW{}, for enumerating all distinct maximal repeats of a string using text indexing arrays such as $SA$ or $BWT$ with auxiliary structures.
After review, in the next section, we propose our novel algorithm schema \TDSA{}.


\textit{Overview of the search strategies}. In what follows, the categories \textsf{BU}/\textsf{TD} indicate \textit{bottom-up/top-down traversal}, respectively, while \textsf{SA}/\textsf{BW} indicate and \textit{SA/BWT arrays}. 
In \cref{fig:fwdstree}, we show the illustration of the search space, the suffix tree for a string $T = \mathtt{\#abcadabcabc\$}$, for algorithm schema in this paper. In the figure, \BUSA{} traverses the black tree of downward arrows (the suffix tree), while  \TDBW{} traverses the red tree (the Winer tree) of upward arrows (suffix links). Our \TDSA{} in \cref{sec:algo} traverses the black tree, but from the root to the leaves top-down, unlike \BUSA{} or \TDBW{}. 

%% %%%%%%%%
%% %%%%
%%%%%%
\begin{figure}[t]
\centering  
\includegraphics[width=1.00\textwidth]{fig/exp1/fwdstree-crop.pdf}
\smallskip
  \caption{Illustration of the search strategies for maximal repeat enumeration. The figure shows two search trees for a string $S = \mathtt{\#abcadabcabc\$}$ of \cref{tbl:arrays}: the black tree is the suffix tree of $S$; the red tree is the Winer tree of $S$. In the trees, each circle indicates a node with a right-branching substring as its string label, each double black-red circle indicates a maximal repeat, and each cross a locus on an edge. To each node, its SA-range $[L..R]$ or $[L]$ in \cref{tbl:arrays} is attached. 
    Note that each black edge has a string label of length one or more, while each red edge has a character label of length one.
    The previous algorithms \BUSA{} and \TDBW{} in in \cref{sec:prev} traverses the entire black subtree of $\Theta(n)$ nodes bottom-up (\cref{rem:lb:busa}) and red subtree of $\Theta(n)$ nodes top-down (\cref{rem:lb:tdbw}), while the proposed \TDSA{} in \cref{sec:algo} traverses only $\mu$ red nodes following $O(e_R)$ black edges (\cref{lem:prune:leftbranch}). $\diamond$
    %% \BUSA{} and \TDSA{} traverse the black subtree bottom-up and top-down, resp., while \TDBW{} traverses the red subtree top-down. 
}\label{fig:fwdstree}
\end{figure}
%%%%%%


%% %%%%%%%%

%%%% 
\subsection{Bottom-up algorithm with SA}
\label{sec:prev:busa}
%%%%%
The first algrithm scheme $\BUSA$ (\textit{bottom-up search with the suffix array}), proposed by Narisawa \textit{et al.}~\cite{narisawa2007efficient,narisawa2017efficient} in 2007, is a procedure that performs the bottom-up search over SA-ranges in forward direction.
Okanohara and Tsujii~\cite{okanohara2009text} adopted a similar scheme.
\footnote{Although it seems that \cite{okanohara2009text} uses $BWT$ at first glance, it essentially belongs to \BUSA{} since it does not use $BWT$ for top-down branching.}
Specifically, it simulates the post-order traversal of the suffix tree for the same text using $SA$, $ISA$, and the RMQ structure on the top of $LCP$ by the \textit{bottom-up traversal method} of
Kasai \textit{et al.}~\cite{kasai:lee2001lcp:linear}.
%% Kasai, Lee, Arimura, Arikawa, and Park~\cite{kasai:lee2001lcp:linear}.
Therefore, in \cref{fig:fwdstree}, \BUSA{} traverse the suffix tree~\cite{gusfield1997algorithms} shown in black bottom up from the leaves to the root, following the reversed arrows (reversed tree edges). 

%%%%%%%%%%%%%%%%%
{
  \setlength{\interspacetitleruled}{0pt}%
  \setlength{\algotitleheightrule}{0pt}%  
  \begin{algorithm}[h]
  \textbf{Procedure} \BUSA$()$:\\
  \Begin{
      $B \gets \set{ (-1,-1) }$; $n \gets |S|$\; 
      \For{$i \gets 2$ \textbf{to} $n+1$}{
        $(L_*, \ell_*) \gets (i-1, LCP[i])$\; 
        \While{$ B.\op{top().depth()} > \ell_* $}{
          $(L, \ell) \gets B.\op{pop()}$; $R \gets i-1$
            \Comment*{$(L, R, \ell)$ is right-branching}
          \If{$(L, R, \ell)$ is left-branching}{
            \textbf{output} $(L, R, \ell)$
            \Comment*{A maximal repeat is found}
          }
          $L_* \gets L$\; 
        }
        \iIf{$B.\op{top().depth()} < \ell_*$}{
          $B.\op{push}((L_*, \ell_*)) $\; 
        }
        $B.\op{push}((i, i, n - SA[i] + 1))$\;
      }
  }%%Begin
  \end{algorithm}
}  
%%%%%%%%%%%%%%%%%

In the procedure \BUSA{}, whenever the triple $\tau = (L, R, \ell)$ is encountered at Line~7, it represents a right-branching substring of a string $S$ because
%it is ensured
the depth $\ell$ equals the range minima of the subarray $LCP[L..R]$ by induction on the stage~$i\ge 1$. Therefore, if the triple $\tau$ at Line~7 is left-branching, we see that it is both left- and right-branching, and thus, a maximal repeat of $S$. By amortized analysis~\cite{kasai:lee2001lcp:linear}, the procedure executes the code for Lines from 7 to 10 at most $O(n)$ time.
As shown in the next section, the test for left-branchingity can be done in $O(1)$ time using $SA$ and $ISA$ only.

On the other hand, \BUSA{} always takes $\Omega(n)$ time since it must visit all right-branching triple, corresponding to all of $\Theta$ internal nodes of the suffix tree. Actually, the for-loop from Lines 4 to 12 is executed exactly $n$ times, and each of them executes at least one instruction. Hence, we observe: 


\begin{remark}[Narisawa \textit{et al.}~\cite{narisawa2007efficient}]
\label{rem:lb:busa}  
%% \begin{remark}[Narisawa, Inenaga, Bannai, and Takeda~\cite{narisawa2007efficient}]
  The algorithm \BUSA{} requires $\Theta(n)$ time to compute all maximal repeats in a string of length $n$, regardless of $e_R \le n$. 
\end{remark}


%%%%%
\subsection{Top-down algorithm with BWT}
\label{sec:prev:tdbw}
%%%%%
The second algrithm scheme $\TDBW$ (\textit{top-down search with the Burrows-Wheeler Transformation/BWT}), proposed by Beller, Berger, and Ohlebusch~\cite{beller:berger2012space:efficient:bbo} 2012, is a recursive procedure that performs
the top-down search over SA-ranges in backward direction using the LF-mapping on BWT,
where the top-down traversal with BWT has been devised by~\cite{bellergogohlebusch2013computing,beller:berger2012space:efficient:bbo,belazzougui2015space:unusual,belazzougui2020linear}. 
Therefore, in \cref{fig:fwdstree}, \TDBW{} traverse the red tree (the Weiner tree) top-down from the root to the leaves following the reversed red links (reversed suffix links or Weiner links~\cite{gusfield1997algorithms}). 

%%%%%%%%%%%%%%%%%
{
  \setlength{\interspacetitleruled}{0pt}%
  \setlength{\algotitleheightrule}{0pt}%  
  \begin{algorithm}[h]
  %% \caption{Top-down MR-enumeration algorithm with BWT}\label{algo:maxrep:tdbw}
  \textbf{Procedure} \TDBW$([L_0..R_0])$:\\
  %%\KwGiven{}
  %% \KwIn{An SA-range $[L_0..R_0]$ for a string.}
  %% \KwOut{}
  \Begin{
      Compute the set $\set{a_1, \dots, a_k}$ of $k$ mutually distinct characters in the subarray $BWT[L_0..R_0] \in \Sigma^*$ and the subranges $[L_1..R_1], \dots, [L_k..R_k]$\;
      \If (\CM{$[L_0..R_0]$ is left-branching}) {$k > 1$}{
        $\ell_0 \gets RMQ_{LCP}(L_k+1, R_k)$
          \Comment*{Making $([L_0..R_0], \ell_0)$ right-branching}
        \textbf{output} $\tau = ([L_0..R_0], \ell_0)$ 
        \Comment*{A maximal repeat is found}
      }
      \For %(\CM{})
          {$i \in [k]$}
      {
         Compute the $i$-th SA-range $[L_i..R_i]$ from $[L_0..R_0]$ by backward search with character $a_i$\;  
        %% Decide if $\tau$ is left-branching by SA and ISA (\cref{lem:leftmaximal:character})\; 
        \TDBW$([L_i..R_i])$\; 
      }%%For 
  }
  \end{algorithm}
}  
%%%%%%%%%%%%%%%%%

The algorithm \TDBW{} starts with the SA-range $[1..n]$ corresponding to the empty string $\eps$ as the root. Suppose that the current pair is $\tau_0 = ([L_0..R_0], \ell_0)$ representing  a substring $U \in Substr(S)$.
Given the SA-range $[L_0..R_0]$, it first computes the sequence of $k\ge 1$ mutually distinct characters $a_1, \dots, a_k \in \Sigma$ such that $a_1 U, \dots a_k U \in Substr(S)$ in $O(1)$ time per character, using the colored range query structure (CRQ) with $O(n)$ space by Muthukrishnan~\cite{muthukrishnan2002efficient}%
\footnote{It is also known as the range distinct query (RDQ) structure.}.
%%% 
Next, it computes the sequence of SA-ranges $[L_1..R_1], \dots, [L_k..R_k]$ for the substrings $a_1 U, \dots a_k U$, respectively, in $O(\log\sigma)$ time per symbol $a_i$ for all $i \in [k]$, using the LF-mapping supported by the Wavelet tree with $BWT$ in $O(n/\log_\sigma n)$ space, by the method of Ferragina and Manzini~\cite{Ferragina05:FM}.

By construction, we can see that this algorithm follows atomic Weiner links reading their character labels one by one. Since it is shown that the Weiner tree for a string of length $n$ contains $\Theta(n)$ atomic Weiner edges. Hence, we conclude: 

%% \begin{remark}[Beller, Berger, and Ohlebusch~\cite{beller:berger2012space:efficient:bbo}]
\begin{remark}[Beller \textit{et al.}~\cite{beller:berger2012space:efficient:bbo}]
\label{rem:lb:tdbw}    
  The algorithm \TDBW{} requires $\Theta(n\log\sigma)$ time to compute all maximal repeats in a string of length $n$, regardless of $e_R \le n$. 
\end{remark}

Belazzougui and Cunial~\cite{belazzougui2015space:unusual} improved the time complexity of the procedure \TDBW{} to run in $O(n)$ time retaining $O(n/\log_\sigma n)$ space by removing $O(\log\sigma)$ factor. However, it still requires the running time $\Theta(n)$. 
