%%%%%% new 

%% notation

\newcommand{\trimbox}[2]{\makebox[#1][l]{${#2}$}}
%% postscriptfont
%\renewcommand{\rmdefault}{ptm} %times -> not used 
\renewcommand{\sfdefault}{phv} %helvetical 
%\renewcommand{\ttdefault}{pcr} %times typewriter type 
\normalfont %default

%% font 
%% \newcommand{\sig}[1]{\mathcal{#1}}
%% \newcommand{\alg}[1]{\idsc{#1}} %% algorithm 
%% \newcommand{\proc}[1]{\alg{#1}}
%% \newcommand{\op}[1]{\texttt{#1}} %% operation 
%% \newcommand{\pb}[1]{\textsc{#1}} %% problem name 
\newcommand{\kw}[1]{\textbf{#1}}
\newcommand{\fun}[1]{\idrm{#1}} %% 
\newcommand{\ul}[1]{\underline{#1}}
%% \newcommand{\prob}[1]{\textsc{#1}}
\newcommand{\name}[1]{\textit{#1}}

%% number math 
\newcommand{\nat}{\mathbb{N}}
\newcommand{\natpos}{\mathbb{N}_{+}}
\newcommand{\rat}{\mathbb{R}}
\newcommand{\ratpos}{\mathbb{R}_{\ge 0}}
\newcommand{\zat}{\mathbb{Z}}
\newcommand{\bat}{\mathbb{B}}
\newcommand{\Bin}{\set{0,1}}
%% \renewcommand{\vec}[1]{\boldsymbol{#1}} %% loaded by amsbmy (amsmath)
\newcommand{\daller}{\$} %$
\newcommand{\ch}[1]{\mbox{`$#1$'}} %引用された文字
\newcommand{\superceil}[1]{\lceil\hskip-0.2em\lceil {#1} \rceil\hskip-0.2em\rceil } 
\newcommand{\lex}{\idrm{lex}}

% math macros %from dsdelight
\newcommand{\sete}[1]{ \{\kern.25em{#1}\kern.25em \}}
\newcommand{\set}[1]{\{\kern0.00em#1\kern0.00em\}} %narroe
\newcommand{\lst}[1]{[\kern0.05em#1\kern0.05em]}
%% \newcommand{\inset}[2]{\{\: #1 \,:\, #2\:\}}
\newcommand{\eps}{\varepsilon}
\newcommand{\by}{\times}
\newcommand{\pair}[1]{\langle #1\rangle}
\newcommand{\ceil}[1]{\lceil #1\rceil}
\newcommand{\floor}[1]{\lfloor #1\rfloor}
\newcommand{\alphabet}[1]{\{#1\}}
\newcommand{\ol}[1]{\overline{#1}}
\newcommand{\bigbar}[1]{\overline{#1}}
\newcommand{\pow}[1]{2^{#1}}

%% 221123 Expectation
\makeatletter
\DeclareRobustCommand\midop[1]{%
  \mathop{\vphantom{#1}\mathpalette\midop@{#1}}\slimits@
}
\newcommand{\midop@}[2]{%
  \vcenter{%
    \sbox\z@{$#1\sum$}%
    \hbox{\resizebox{\ifx#1\displaystyle0.475\fi\dimexpr\ht\z@+\dp\z@}{!}{$\m@th#2$}}%
    %% \hbox{\resizebox{\ifx#1\displaystyle0.8\fi\dimexpr\ht\z@+\dp\z@}{!}{$\m@th#2$}}%
  }%
}
\makeatother
%% 


\newcommand{\iffdef}{\stackrel{\textrm{def}}{\iff}}
\renewcommand{\iff}{\:\Leftrightarrow\:}

\newcommand{\rk}[1]{^{(#1)}} %%alias 
\newcommand{\idx}[3]{_{#1=#2}^{#3}} %%alias
\newcommand{\btw}[3]{{#2}\le {#1}\le {#3}} %%alias


%% %%%%
%%  \newcommand{\myalgotext}[1]{\Statex \hspace{-0.9\leftmargin}\kw{#1}:{}}
%% \newcommand{\Foreach}[1]{\For{\kw{each} #1}}
%% \newcommand{\ForLine}[1]{\State \kw{for} {#1} \kw{do}\,{}}
%% \newcommand{\EndForLine}{}
%% \newcommand{\IfLine}[1]{\State \kw{if} {#1} \kw{then}\,{}}
%% \newcommand{\EndIfLine}{}
%% \newcommand{\StateAnd}{\:{and}\;\;}


%% \newcommand{\ForInline}[1]{\State \kw{for} {#1} \kw{do}\,{}}
%% \newcommand{\EndForInline}{}
%% \newcommand{\IfInline}[1]{\State \kw{if} {#1} \kw{then}\,{}}
%% \newcommand{\ElseIfInline}[1]{\State \kw{else if} {#1} \kw{then}\,{}}
%% \newcommand{\ElsIfInline}[1]{\ElseIfInline{#1}}
%% \newcommand{\ElseInline}{\State \kw{else} \,{}}
%% \newcommand{\EndIfInline}{}
%% \newcommand{\StateInline}{}


%%%%%%%%%%%%%%%%%%%%%%%%%%%%%%%%%%%%%%%%%%%%%%%%%
%%% apxproof

% \newtheoremrep{lemma}[theorem]{Lemma}
% \newtheoremrep{theorem}[theorem]{Theorem}
% \newtheoremrep{corollary}[theorem]{Corollary}
% \newtheoremrep{remark}[theorem]{Remark}

%% % For proof.
%% \def\lowerbox{\raise-.12ex\hbox{$\Box$}}
%% \newcommand{\qed}{\hfill{\lowerbox}\vspace{0.8\topsep}\par}
%% %\newcommand{\lowerbox}{\hbox{\rule{6pt}{6pt}}} %by Inago-kun
%% \newenvironment{proof}{
%%   %\vspace{0.5\topsep}
%%   \noindent{\textit{Proof}:\rule{.15em}{0mm}}
%% }{\qed } %proof
%%%%%%%%%%%%%%%%%%%%%%%%%%%%%%%%%%%%%%%%%%%%%%%%

\newenvironment{bigpar}{\!\left(}{\right)}
\newenvironment{bigbrack}{\!\left[}{\right]}
\newenvironment{bigbrace}{\!\left\{}{\right\}}

%% bigoper with limit: beg %%%%%%%%%%
%% %%% convolution product 
%% from tex.stackexchange.com: "How t create my own math operator with limits?"
%% https://tex.stackexchange.com/questions/23432/how-to-create-my-own-math-operator-with-limits
%% \DeclareMathOperator*{\convtimes}{\circledast}
\makeatletter
\DeclareRobustCommand\bigop[1]{%
  \mathop{\vphantom{#1}\mathpalette\bigop@{#1}}\slimits@
}
%% \DeclareRobustCommand\bigop[1]{%
%%   \mathop{\vphantom{\sum}\mathpalette\bigop@{#1}}\slimits@
%% }
\newcommand{\bigop@}[2]{%
  \vcenter{%
    \sbox\z@{$#1\sum$}%
    \hbox{\resizebox{\ifx#1\displaystyle0.8\fi\dimexpr\ht\z@+\dp\z@}{!}{$\m@th#2$}}%
  }%
}
%%%
\makeatother
%% bigoper with limit: end %%%%%%%%%%
%% bigoper with limit: beg %%%%%%%%%%
%%% mybigopmid:x0.7
\makeatletter
\DeclareRobustCommand\mybigopin[1]{%
  \mathop{\vphantom{\sum}\mathpalette\mybigopin@{#1}}\slimits@
}
\newcommand{\mybigopin@}[2]{%
  \vcenter{%
    \sbox\z@{$#1\sum$}%
    \hbox{\resizebox{\ifx#1\displaystyle1.0\fi\dimexpr\ht\z@+\dp\z@}{!}{$\m@th#2$}}%
%    \hbox{\resizebox{\ifx#1\displaystyle0.9\fi\dimexpr\ht\z@+\dp\z@}{!}{$\m@th#2$}}%    
  }%
}
%%
\DeclareRobustCommand\mymidopin[1]{%
  \mathop{\vphantom{\sum}\mathpalette\mymidopin@{#1}}\slimits@
}
\newcommand{\mymidopin@}[2]{%
  \vcenter{%
    \sbox\z@{$#1\sum$}%
    \hbox{\resizebox{\ifx#1\displaystyle0.25\fi\dimexpr\ht\z@+\dp\z@}{!}{$\m@th#2$}}%
  }%
}
\makeatother
%% bigoper with limit: end %%%%%%%%%%

%% usage
\newcommand{\bigzero}{\DOTSB\mybigopin{\mathbb{O}}\strut}
\newcommand{\midzero}{\mathbb{O}}

%\newcommand{\bigdigit}[1]{\DOTSB\mybigop{\mathbbm{#1}}\strut}
%\newcommand{\mymiddigit}[1]{\DOTSB\mymidop{\mathbbm{#1}}\strut}
\newcommand{\mybigop}[1]{\DOTSB\mybigopin{\mathbbm{#1}}\strut}
\newcommand{\mymidop}[1]{\DOTSB\mymidopin{\mathbbm{#1}}\strut}

\newcommand{\bigsym}[1]{\mybigop{#1}\strut}
\newcommand{\midsym}[1]{\mymidop{#1}\strut}


%\newcommand{\middigit}[1]{\mathbbm{#1}}
\newcommand{\ksub}[1]{\strut_{#1}}
%% 

