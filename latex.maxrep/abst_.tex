%% abst_draft.txt

本研究では,unusual wordsと呼ばれるa stringの中の特徴的なパターンの列挙問題を考察する.一般に,ある文字列中の部分語が unusual wordであるとは,そのすべてのproper substringsの出現回数が,それ自身の出現回数よりも真に大きいときをいう.Unusual words の例として,minimal rare words (MRW), minimal unique substrings (MUSs), minimal absent words (MAW)等の文字列パターンのクラスが提案され,応用のbioinformatics や 系列解析の分野で,それらの列挙アルゴリズムが盛んに研究されている.これらのalgorithmsは,suffix arrayやBurrows-Wheeler transform 等のstring index arrays上で the suffix treeを模倣することで,実用的な計算時間と領域効率を達成している.これらのパターンの総数$M$は入力文字列に依存して大きく変化するが,出力サイズ$M$が入力文字列の長さ$n$より著しく小さい場合も,既存のアルゴリズムは常に$\Theta(n)$時間を要し,計算時間が出力依存ではないという問題をもつ.
本論文では,上記のunusual wordsのクラスに対する出力依存な列挙アルゴリズムを考察する.はじめに,a characterization of MRWs in terms of maximal repeats in a stringを与える.次に,このcharacterizationに基づいて,the class $MRW(S)$ of minimal rare wordsに対して,一本のstring中のパターンの列挙のための,出力依存な時間計算量をもつアルゴリズムをpresentする.このアルゴリズムはすべての$occ$個の異なりパターンを $O(e_R + e_L + occ)$時間と$O(e_L + \idrm{min})$時間で重複なしに列挙する.ここに,$e_L$ and $e_R$ denote the numbers of left- and right-extensions of maximal repeats in $S$, respectively, and are upperbounded by the length $n$ of a string, and can be sublinear in $n$, up to logarithmic, for highly repetitive strings such as collection of human genomes.
提案アルゴリズムのために,全ての異なり maximal repeats in a string を,接尾辞配列と, LCP配列, BWT配列を用いて $O(e_R + e_L + occ)$時間と$O(e_L + \idrm{min})$時間で重複なしに列挙する新しいアルゴリズムを開発する.提案アルゴリズムは,従来の索引配列を用いた列挙アルゴリズムの計算時間を改善し,これらのクラスの異なりパターンを準線形時間で列挙する初めての結果である.
