%%%%
%%%%%%
\begin{figure}[t]
\centering  
\includegraphics[width=1.00\textwidth]{fig/exp1/fwdstree-crop.pdf}
\smallskip
  \caption{Illustration of the search strategies for maximal repeat enumeration. The figure shows two search trees for a string $S = \mathtt{\#abcadabcabc\$}$ of \cref{tbl:arrays}: the black tree is the suffix tree of $S$; the red tree is the Winer tree of $S$. In the trees, each circle indicates a node with a right-branching substring as its string label, each double black-red circle indicates a maximal repeat, and each cross a locus on an edge. To each node, its SA-range $[L..R]$ or $[L]$ in \cref{tbl:arrays} is attached. 
    Note that each black edge has a string label of length one or more, while each red edge has a character label of length one.
    The previous algorithms \BUSA{} and \TDBW{} in in \cref{sec:prev} traverses the entire black subtree of $\Theta(n)$ nodes bottom-up (\cref{rem:lb:busa}) and red subtree of $\Theta(n)$ nodes top-down (\cref{rem:lb:tdbw}), while the proposed \TDSA{} in \cref{sec:algo} traverses only $\mu$ red nodes following $O(e_R)$ black edges (\cref{lem:prune:leftbranch}). $\diamond$
    %% \BUSA{} and \TDSA{} traverse the black subtree bottom-up and top-down, resp., while \TDBW{} traverses the red subtree top-down. 
}\label{fig:fwdstree}
\end{figure}
%%%%%%

